\chapter{Substructural logic}

Linear logic is the most famous of the {\it substructural logics}.
Persistent logic admits the three so-called {\it structural rules} of
weakening (premises need not be used), contraction (premeses may be
used multiple times) and exchange (the ordering of premises are 
irrelevant). Substructural logics, then, are logics that do not admit
these structural rules -- linear logic has only exchange, 
{\it affine} logic (which is frequently conflated with linear logic
by programming language designers) has exchange and weakening, and
{\it ordered} logic, first investigated as a proof theory by Lambek
\cite{lambek58mathematics}, lacks all three. 

Calling logics like linear, affine, and ordered logic 
substructural relative to persistent logic, which is structural
is arguably unfair to the substructural logics. Girard's linear
logic can express persistent provability using the exponential
connective ${!}A$, and this idea is generally applicable in
substructural logics, as observed by Polakow and Pfenning who
applied it to Lambek's ordered logic \cite{}.

it was applied to Lambek's ordered
logic)

 who can generally
recover all the power of persistent logic through the use of 
the exponential connective ${!}A$. 

The name ``substructural logic'' arguably gives short schrift to the
these logics, which 


Another name that would work
is {\it structural} logic. In formal presentations of logic, this
persistence manifests itself as the so-called {\it structural
  properties} of hypothetical reasoning -- {\it weakening} (which
allows hypotheses to go unused), {\it contraction} (which allows
hypotheses to be duplicated for reuse), and {\it exchange} (which
enforces that the ordering of hypotheses is not meaningful)
\cite{gentzen35untersuchungen}.

\section{Ordered linear logic}

Ordered linear logic 

\begin{figure}
\small
{\it Atomic propositions}
\[
\infer[{\it init}^+]
{\oiseq{\Gamma}{p}}
{}
\]

\medskip
{\it Modalities}
\[
\infer[{\gnab}_R]
{\orseq{\Gamma}{\Delta}{\cdot}{{\gnab}A}}
{\otseq{\Gamma}{\Delta}{\cdot}{A}}
\quad
\infer[{\gnab}_L]
{\olseq{\Gamma}{\Delta}{\Omega_L}{{\gnab}A}{\Omega_R}}
{\opseq{\Gamma}{\Delta, A}{\Omega_L}{\Omega_R}}
\]
\vspace{-5pt}
\[
\infer[{!}_R]
{\orseq{\Gamma}{\cdot}{\cdot}{{!}A}}
{\otseq{\Gamma}{\cdot}{\cdot}{A}}
\quad
\infer[{!}_L]
{\olseq{\Gamma}{\Delta}{\Omega_L}{{!}A}{\Omega_R}}
{\opseq{\Gamma,A}{\Delta}{\Omega_L}{\Omega_R}}
\]
\vspace{-5pt}
\[
\infer[{\ocircle}_R]
{\orseq{\Gamma}{\Delta}{\Omega}{{\ocircle}A}}
{\oseq{\Gamma}{\Delta}{\Omega}{\islax{A}}}
\quad
\infer[{\ocircle}_L]
{\oseq{\Gamma}{\Delta}{\Omega_L /{\ocircle}A/ \Omega_R}{\islax{C}}}
{\oseq{\Gamma}{\Delta}{\Omega_L, A, \Omega_R}{\islax{C}}}
\]

\medskip
{\it Multiplicative connectives}
\[
\infer[{\one}_R]
{\orseq{\Gamma}{\cdot}{\cdot}{\one}}
{}
\quad
\infer[{\one}_L]
{\olseq{\Gamma}{\Delta}{\Omega_L}{\one}{\Omega_R}}
{\opseq{\Gamma}{\Delta}{\Omega_L}{\Omega_R}}
\]
\[
\infer[{\fuse}_R]
{\orseq{\Gamma}{\Delta_1,\Delta_2}{\Omega_L,\Omega_R}{A \fuse B}}
{\otseq{\Gamma}{\Delta}{\Omega_L}{A}
 &
 \otseq{\Gamma}{\Delta}{\Omega_R}{B}}
\quad
\infer[{\fuse}_L]
{\olseq{\Gamma}{\Delta}{\Omega_L}{A \fuse B}{\Omega_R}}
{\opseq{\Gamma}{\Delta}{\Omega_L}{A,B,\Omega_R}}
\]
\vspace{-5pt}
\[
\infer[{\righti}_R]
{\orseq{\Gamma}{\Delta}{\Omega}{A \righti B}}
{\otseq{\Gamma}{\Delta}{\Omega, A}{B}}
\quad
\infer[{\righti}_L]
{\olseq{\Gamma}{\Delta_A, \Delta}{\Omega_L}{A \righti B}{\Omega_A, \Omega_R}}
{\otseq{\Gamma}{\Delta_A}{\Omega_A}{A}
 &
 \opseq{\Gamma}{\Delta}{\Omega_L}{B, \Omega_R}}
\]
\vspace{-5pt}
\[
\infer[{\lefti}_R]
{\orseq{\Gamma}{\Delta}{\Omega}{A \lefti B}}
{\otseq{\Gamma}{\Delta}{A, \Omega}{B}}
\quad
\infer[{\lefti}_L]
{\olseq{\Gamma}{\Delta_A, \Delta}{\Omega_L, \Omega_A}{A \lefti B}{\Omega_R}}
{\otseq{\Gamma}{\Delta_A}{\Omega_A}{A}
 &
 \opseq{\Gamma}{\Delta}{\Omega_L}{B,\Omega_R}}
\]

\medskip
{\it Additive connectives}
\[
\infer[{\zero}_L]
{\olseq{\Gamma}{\Delta}{\Omega_L}{\zero}{\Omega_R}}
{}
\quad
\infer[{\oplus}_{R1}]
{\orseq{\Gamma}{\Delta}{\Omega}{A \oplus B}}
{\otseq{\Gamma}{\Delta}{\Omega}{A}}
\quad
\infer[{\oplus}_{R2}]
{\orseq{\Gamma}{\Delta}{\Omega}{A \oplus B}}
{\otseq{\Gamma}{\Delta}{\Omega}{B}}
\]
\vspace{-5pt}
\[
\infer[{\oplus}_{L}]
{\olseq{\Gamma}{\Delta}{\Omega_L}{A \oplus B}{\Omega_R}}
{\opseq{\Gamma}{\Delta}{\Omega_L}{A,\Omega_R}
 &
 \opseq{\Gamma}{\Delta}{\Omega_L}{B,\Omega_R}}
\]
\vspace{-5pt}
\[
\infer[{\top}_R]
{\orseq{\Gamma}{\Delta}{\Omega}{\top}}
{}
\quad
\infer[{\with}_{L1}]
{\olseq{\Gamma}{\Delta}{\Omega_L}{A \with B}{\Omega_R}}
{\opseq{\Gamma}{\Delta}{\Omega_L}{A, \Omega_R}}
\quad
\infer[{\with}_{L2}]
{\olseq{\Gamma}{\Delta}{\Omega_L}{A \with B}{\Omega_R}}
{\opseq{\Gamma}{\Delta}{\Omega_L}{B, \Omega_R}}
\]
\vspace{-5pt}
\[
\infer[{\with}_R]
{\orseq{\Gamma}{\Delta}{\Omega}{A \with B}}
{\otseq{\Gamma}{\Delta}{\Omega}{A}
 &
 \otseq{\Gamma}{\Delta}{\Omega}{B}}
\]


\caption{Propositional ordered linear lax logic.}
\label{fig:ordered-prop}
\end{figure}




\section{Substructural contexts}

We will treat {\it substructural contexts} $\Delta$ are maps from {\it
  variables} to {\it judgments}. There are two kinds of variables: 
those that map {\it extent}.

${\mbox{\it [}}$


Two of the fundamental propeties of $\Delta$ are 
$x{:}\langle A^+ \rangle_l \sqsubseteq \Delta$. In linear logic, this
is the statement that $x{:}A \in \Gamma$, whereas in 

\[\small
\begin{array}{|c|c|c|c|}
%
\begin{array}{c}
\makebox[1.8in]{\it Dual intuitionstic linear logic}\medskip
\\
\infer
{\Gamma; p^+ \vdash p^+ \mathstrut}
{\mathstrut}
\end{array}
%
&
%
\begin{array}{c}
\makebox[1.8in]{\it Pfenning-Davies S4}\medskip
\\
\infer
{\Delta; \Gamma, p^+ \mathstrut \vdash p^+}
{\mathstrut}
\end{array}
%
&
%
\begin{array}{c}
\makebox[1.8in]{\it Unified}\medskip
\\
\infer
{\Delta \vdash [ p^+ ]_l \mathstrut}
{x{:}\langle p^+ \rangle_l \subseteq \Delta \mathstrut}
\end{array}
%
\end{array}
\]

\[\small
\begin{array}{|c|c|c|c|}
%
\begin{array}{c}
\makebox[1.8in]{\it Dual intuitionstic linear logic}\medskip
\\
\infer
{\Gamma; \Delta, A \with B \vdash C \mathstrut}
{\Gamma; \Delta, A \vdash C \mathstrut}
\end{array}
%
&
%
\begin{array}{c}
\makebox[1.8in]{\it Pfenning-Davies S4}\medskip
\\
\infer
{\Delta; \Gamma, A \wedge^- B \mathstrut \vdash C}
{\Delta; \Gamma, A \wedge^- B, A \vdash C \mathstrut}
\end{array}
%
&
%
\begin{array}{c}
\makebox[1.8in]{\it Unified}\medskip
\\
\infer
{\Theta\{ x{:} A \with B \} \vdash U \mathstrut}
{\Theta\{ x_1{:} A \} \vdash U \mathstrut}
\end{array}
%
\end{array}
\]


\[\small
\begin{array}{|c|c|c|c|}
%
\begin{array}{c}
\makebox[1.8in]{\it Dual intuitionstic linear logic}\medskip
\\
\infer
{\Gamma; \Delta_1, \Delta_2 \vdash A \otimes B \mathstrut}
{\Gamma; \Delta_1 \vdash A & \Gamma; \Delta_2 \vdash B \mathstrut}
\end{array}
%
&
%
\begin{array}{c}
\makebox[1.8in]{\it Pfenning-Davies S4}\medskip
\\
\infer
{\Delta; \Gamma \vdash A \wedge^+ B \mathstrut}
{\Delta; \Gamma \vdash A & \Delta; \Gamma \vdash B \mathstrut}
\end{array}
%
&
%
\begin{array}{c}
\makebox[1.8in]{\it Unified}\medskip
\\
\infer
{\Delta_1 \bowtie \Delta_2 \vdash A \otimes B \mathstrut}
{\Delta \vdash A & \Delta_2 \vdash B \mathstrut}
\end{array}
%
\end{array}
\]



\section{Explicit shifts in focusing}

\section{Polarization and erasure}

\section{Cut admissibility}

\section{Identity expansion}

\section{Unfocused admissibility}

\section{Soundness and completeness}

\section{Ordered linear logic}
