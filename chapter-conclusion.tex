\chapter{Conclusion}
\label{chapter-conclusion}

In this thesis, we have developed a logical framework of substructural
logical specifications, \sls, based on a rewriting interpretation of
ordered linear lax logic (\ollll). Part I discussed the design of this
logical framework, and in the process firmly established the
connection between 1) canonical forms and hereditary substitution in a
logical framework and 2) focused derivations and cut admissibility in
logic. The existance of this connection has been known for a decade,
but this thesis gives the first account of the connection that
generalizes to all logical connectives. This connection allowed the
\sls~framework to be presented as a syntactic refinement of focused
ordered linear lax logic; the steps and traces of \sls, which provide
its rewriting interpretation, are justified as partial proofs in
focused ordered linear lax logic.

The \sls~framework is intended to act as a bridge between the world of
logical frameworks, where deductive derivations are the principal
objects of study, and the world of rewriting logic, where artifacts
similar to \sls~traces are the principal objects of study. Part II of
this thesis discusses a number of ways of describing operational
semantics specifications in \sls, using ordered resources to encode
control structures, using mobile/linear resources to encode mutable
state and concurrent communication, and using persistent resources to
represent memoization and binding. \sls~still hews more closely to the
tradition of logical frameworks, but future work will hopefully reduce
the remaining distance between operational semantics specifications in
\sls~and rewriting logic-based approaches to operational semantics
specifications.

The approximation methodology presented in
Chapter~\ref{chapter-approx} is also relevant to this refinement of
the thesis statement. Program abstraction is also a form of formal
reasoning about the properties of the programming language. One the
other hand, approximation is a program transformation like
operationalization, defunctionalization, and destination adding, which
is why this chapter was presented in Part~II. This observation
illustrates that Chapter~\ref{chapter-approx} sits uncertainly between
Part~II and Part~III of this thesis.)

