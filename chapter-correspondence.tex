\chapter{On logical correspondence}

In Part 1, we defined \sls, the logical framework of substructural
logical specifications.

\section{Logical transformation: compilation}

\subsection{Tail-recursion}

\subsection{Parallelism}

\section{Logical transformation: defunctionalization}

\section{Logical transformation: factoring}

Example: exceptions

\section{Exploring the richer fragment}

\subsection{Mutable storage}
\label{sec:mutable-storage}

No check for pointer inequality! This is a fundamental restriction of
the fact that we're using existential quantificaiton rather than some
form of nominal quantification. (Hack due to Favonia and Bob, personal
communication.)

\subsection{Call-by-need}

\subsection{Environment semantics}

\subsection{Looking back at natural semantics}
\label{sec:enriching-natsem}

\section{Partial transformation}

\subsection{Evaluation contexts}

\subsection{Temporal logic}

The natural semantics of \rowan~are not, on a superficial level,
significantly more complex than other natural semantics. However, it
turns out that the usual set of techniques for adding state to a
natural semantics break down, and discussing a \rowan-like logic with
state remained a challenge for many years.\robnote{Figure out from
  Rowan what the recent work he told you about was.} Through the
logical correspondance, it is easy to see why: the natural SSOS
specification of \rowan~integrates both concurrent and deductive
reasoning in an arbitrarily nested way. In fact, Figure XXX is the
only SLS specification in this thesis that exhibits this form of
recursive dependency between concurrent and deductive reasoning.  In
particular, the \rowan~specification is way out of the image of the
extended natural semantics we considered in
Section~\ref{sec:enriching-natsem}. The natural encoding in state lies
in the ambient substructural context of a concurrent computation, but
that ambient computation cannot properly enter into a deductive
sub-computation. If we tried to add state to \rowan~the same way we
added it in Section~\ref{sec:mutable-storage}, the entire store
would effectively leave scope whenever computation considered
the subterm $e$ of ${\sf next}(e)$. That consideration happens
as deductive reasoning, not as concurrent reasoning!

 it is the only we
will consider in this thesis that has with property.

It's hard to include state in temporal logic! But the logical correspondence
helps us understand why: the natural SSOS specification of 