
\chapter{Linear logic}

Logic as it has been traditionally understood and studied -- both in
its classical and intuitionistic varieties -- treats the truth of a
proposition as a {\it persistent resource}. That is, if we have
evidence for the truth of a proposition, we can ignore that evidence
if it is not needed and reuse the evidence as many times as we need
to. Throughout this thesis, ``logic as it has been traditionally
understood as studied'' will be referred to as {\it persistent} logic
to emphasize this treatment of evidence. 

Linear logic, which was popularized by Girard \cite{girard87linear},
treats evidence as an {\it ephemeral} resource, which is generally
consumed when it is used. Linear logic, like persistent logic, comes
in classical and intuitionistic flavors. We will favor intuitionistic
linear logic in part because the propositions of intuitionistic linear
logic (written $A$, $B$, $C$, \ldots) have a more natural
correspondence with physical resources. Linear conjunction $A \tensor
B$ represents the resource built from the resources $A$ and $B$; if
you have both a bowl of soup {\it and} a sandwich, that resource can
be represented by the proposition ${\sf soup} \otimes {\sf
  sandwich}$. Linear implication $A \lolli B$ represents a resource
that can interact with another resource $A$ to produce a resource
$B$. A robot with batteries not included could be represented as the
resource $({\sf batteries} \lolli {\sf robot})$, and the linear resource
$({\sf dollars} \lolli {\sf soup} \tensor {\sf sandwich})$ represents
the ability to use some dollars to obtain lunch -- but only
once!\footnote{Conjunction will always bind more tightly than
  implication, so this is equivalent to the proposition ${\sf dollars}
  \lolli ({\sf soup} \tensor {\sf sandwich})$.} Linear logic also has
a modality ${!}A$ representing a resource that can be used to generate
any number of $A$ resources, including zero. The Panera ``You Pick
Two'' menu might be represented as
\[ {!}({\sf dollars} \lolli {\sf soup} \tensor {\sf sandwich}) \otimes
{!}({\sf dollars} \lolli {\sf soup} \tensor {\sf salad}) \otimes
{!}({\sf dollars} \lolli {\sf sandwich} \tensor {\sf salad}),\] as the
menu gives you the opportunity to exchange dollars for two of the three
components any number of times.

\begin{figure}[t]
\begin{tabbing}
\quad $A$ \,\, \=  
   $::= p \mid {!}A \mid \one \mid A \tensor B \mid A \lolli B$\\
\quad $\Gamma$ \> $::= \cdot \mid \Gamma, A$ \qquad \= {\it (multiset)}\\
\quad $\Delta$ \> $::= \cdot \mid \Delta, A$ \> {\it (multiset)}\\
\end{tabbing}
%
%
\quad \fbox{$\seq{\Gamma}{\Delta}{A}$}
\[
\infer[{\it id}]
{\seq{\Gamma}{p}{p}}
{}
\qquad
\infer[{\it copy}]
{\seq{\Gamma, A}{\Delta}{C}}
{\seq{\Gamma, A}{\Delta, A}{C}}
%
\]

\[
%
\infer[{!}_R]
{\seq{\Gamma}{\cdot}{{!}A}}
{\seq{\Gamma}{\cdot}{A}}
\qquad
\infer[{!}_L]
{\seq{\Gamma}{\Delta, {!}A}{C}}
{\seq{\Gamma, A}{\Delta}{C}}
\qquad
\infer[\one_R]
{\seq{\Gamma}{\cdot}{\one}}
{}
\qquad
\infer[\one_L]
{\seq{\Gamma}{\Delta, \one}{C}}
{\seq{\Gamma}{\Delta}{C}}
\]

\[
%
\infer[{\tensor}_R]
{\seq{\Gamma}{\Delta_1,\Delta_2}{A \tensor B}}
{\seq{\Gamma}{\Delta_1}{A}
 &
 \seq{\Gamma}{\Delta_2}{B}}
\qquad
\infer[{\tensor}_L]
{\seq{\Gamma}{\Delta, A \tensor B}{C}}
{\seq{\Gamma}{\Delta, A, B}{C}}
\]

\[
%
\infer[{\lolli}_R]
{\seq{\Gamma}{\Delta}{A \lolli B}}
{\seq{\Gamma}{\Delta, A}{B}}
\qquad
\infer[{\lolli}_L]
{\seq{\Gamma}{\Delta_1,\Delta_2, A \lolli B}{C}}
{\seq{\Gamma}{\Delta_1}{A}
 &
 \seq{\Gamma}{\Delta_2, B}{C}}
%
\]
\caption{Intuitionstic linear logic}
\label{fig:linear}
\end{figure}


Figure~\ref{fig:linear} presents a formal system for linear logic, in
particular the so-called {\it multiplicative, exponential} fragment of
intuitionistic linear logic (or {\it MELL}). There is one quirk to
this presentation: I've presented two identical rules for atomic
propositions, $p^+$ and $p^-$, and given two identitial rules for using
them, ${\it init}^+$ and ${\it init}^-$.

There is one unusual aspect to this presentation: 
There is one unusual aspect to the presentation in
Figure~\ref{fig:linear} relative to most presentations of linear
logic, which is that we define two separate classes of atomic
propositions, $p^+$ and $p^-$, and give two identitial rules for using
them, ${\it init}^+$ and ${\it init}^-$. Other than that,
Figure~\ref{fig:linear} is a standard presentation of the so-called
{\it multiplicative, exponential} fragment of intuitionistic linear
logic (or {\it MELL}) in Figure~\ref{fig:linear}. It corresponds to
Andreoli's dyadic system \cite{andreoli92logic}, Barber's dual
intuitionistic linear logic \cite{barber96dual}, and Chang et al.'s
judgmental analysis of intuitionistic linear logic
\cite{chang03judgmental}.  

The propositions of linear logic, and linear implication in
particular, capture a notion of state change: we can {\it transition}
from a state where we have both ${\sf batteries}$ and the powered-off
robot (represented, as before, by the proposition ${\sf batteries}
\lolli {\sf robot}$) to a state where we have the powered on robot
(represented by the proposition ${\sf robot}$). In other words, the
proposition
%
\[{\sf batteries} \otimes ({\sf batteries} \lolli {\sf robot}) \lolli
{\sf robot}\] 
%
is provable in linear logic. These transitions can be chained
together in a natural way as well: if we start out with ${\sf
  dollars}$ instead of ${\sf batteries}$ but we also have the
persistent ability to turn ${\sf dollars}$ into ${\sf batteries}$
(just like we turned them into a bowl of soup and a salad at Panera),
then we can ultimately get our working robot as well. This boils down
to the statement that it is possible to prove
\[{!}({\sf dollars} \lolli {\sf batteries}) \otimes {\sf dollars}
\otimes ({\sf batteries} \lolli {\sf robot}) \lolli {\sf robot}\] in
linear logic; a derivation of this proposition is given in
Figure~\ref{fig:unfocused-robot}.\footnote{In Chapter XXX, I will
  argue that this view isn't quite precise enough, and that the most
  natural representation of state change from the state $A$ to the
  state $B$ isn't really captured by derivations of the proposition $A
  \lolli B$ or by derivations of the hypothetical judgment
  $\seq{\cdot}{A}{B}$.  However, this view remains a simple and useful
  one; Cervesato and Scedrov cover it thoroughly in the context of
  intuitionstic linear logic \cite{cervesato09relating}.}

If we have ${\sf dollars}$ instead of 
${\sf batteries}$ but the persistent context $\Gamma$ contains the 
persistent proposition $({\sf dollars} \lolli {\sf batteries})$, we can
transition, with a little bit more work

We view see the following derivation, which
follows the rules for linear logic given in Figure~\ref{fig:linear},
as justifying or witnessing that transition:
\begin{figure}
\[
\infer[{\lolli}L]
{\seq{\cdot}{\cdot}{{!}({\sf dollars} \lolli {\sf batteries}) \otimes
                    {\sf dollars} \otimes 
                    ({\sf batteries} \lolli {\sf robot}) \lolli {\sf robot}}}
{\infer[{\otimes}L]
{\seq{\cdot}{{!}({\sf dollars} \lolli {\sf batteries}) \otimes
                    {\sf dollars} \otimes 
                    ({\sf batteries} \lolli {\sf robot})}{{\sf robot}}}
{\infer[{!}L]
{\seq{\cdot}{{!}({\sf dollars} \lolli {\sf batteries}),
                    {\sf dollars} \otimes 
                    ({\sf batteries} \lolli {\sf robot})}{{\sf robot}}}
{\infer[{\otimes}L]
{\seq{\Gamma}{{\sf dollars} \otimes 
                    ({\sf batteries} \lolli {\sf robot})}{{\sf robot}}}
{\infer[{\lolli}L]
{\seq{\Gamma}{{\sf dollars}, {\sf batteries} \lolli {\sf robot}}{{\sf robot}}}
{\infer[{\it copy}]
 {\seq{\Gamma}{{\sf dollars}}{{\sf batteries}}}
 {\infer[{\lolli}L] 
  {\seq{\Gamma}{{\sf dollars}, {\sf dollars} \lolli {\sf batteries}}{{\sf batteries}}}
  {\infer[{\it init}^+]
   {\seq{\Gamma}{{\sf dollars}}{{\sf dollars}}}
   {}
   &
   \infer[{\it init}^+]
   {\seq{\Gamma}{{\sf batteries}}{{\sf batteries}}}
   {}}}
 &
 \infer[{\it init}^+]
 {\seq{\Gamma}{{\sf robot}}{{\sf robot}}}
 {}}}}}}
\] 
\caption{Proving that a transition is possible 
(where we let $\Gamma = {\sf dollars} \lolli {\sf batteries}$).}
\label{fig:unfocused-robot}
\end{figure}
It is precisely because linear logic contains this natural notion of
state and state transition that a rich line of work, dating back to
Chirmar's Ph.D. thesis, has sought to use linear logic as a {\it
  logical framework} for describing stateful systems
\cite{chirimar95proof,cervesato02linear,
  cervesato02concurrent,pfenning04substructural,miller09formalizing,
  pfenning09substructural,cervesato09relating}.

While linear logic has many uses in computer science, I'll be
exclusively concentrating on its use as a logical framework for
describing stateful systems. Logical frameworks use the {\it
  structure} of proofs in a logic to describe

 To that end, I will describe 

Linear logic, in both its intuitionistic and classical variants, has
many uses throughout logic and computer science. While this thesis
will make some general observations about, and contributions to, the
proof theory of intuitionistic substructural logic,\footnote{Linear logic is
just one of the many substructural logics, a point we will return to
later.} I do so in pursuit of a very specific goal: I want to {\it encode
stateful evolving systems in substructural logic.}

\section{Focused logic and synthetic inference rules}


 This can be phrased
in one of two ways: we can say that there are now three judgment
forms:
\begin{itemize}
\item $\mildrfoc{\Gamma}{\Delta}{C}$ (the {\it right focus} judgment),
\item $\mildinv{\Gamma}{\Delta}{C}$ (the {\it inversion} judgment), and
\item $\mildlfoc{\Gamma}{\Delta}{A}{C}$ (the {\it left focus} judgment).
\end{itemize}

\[
\infer[{\it focus}R]
{\seq{\Gamma}{\Delta}{A^+}}
{\seq{\Gamma}{\Delta}{[A^+]}}
\qquad
\infer[{\it focus}L]
{\seq{\Gamma}{\Delta,A^-}{C}}
{\seq{\Gamma}{\Delta,[A^-]}{C}}
\]\[
\infer[{\it blur}R]
{\seq{\Gamma}{\Delta}{[A^-]}}
{\seq{\Gamma}{\Delta}{A^-}}
\qquad
\infer[{\it blur}L]
{\seq{\Gamma}{\Delta, [A^+]}{C}}
{\seq{\Gamma}{\Delta, A^+}{C}}
\]


The story of {\it focusing} in linear logic is not quite so 
obviously settled, however, especially when it comes to the treatment
of atoms. To try and explain why, we will first consider at 
an intuitionistic system that is, I believe, faithfully adapted from 
Andreoli's original presentation of linear logic; this system is
presented in Figure~\ref{fig:focused}.

\section{Positive atomic propositions}

\chapter{Substructural logic}

Another name that would work
is {\it structural} logic. In formal presentations of logic, this
persistence manifests itself as the so-called {\it structural
  properties} of hypothetical reasoning -- {\it weakening} (which
allows hypotheses to go unused), {\it contraction} (which allows
hypotheses to be duplicated for reuse), and {\it exchange} (which
enforces that the ordering of hypotheses is not meaningful)
\cite{gentzen35untersuchungen}.
