
\chapter{Linear logic}

Logic as it has been traditionally understood and studied -- both in
its classical and intuitionstic varities -- treats the truth of a
proposition as a {\it persistent resource}. That is, if we have
evidence for the truth of a proposition, we can ignore that evidence
if it is not needed and reuse the evidence as many times as we need
to. Throughout this thesis, ``logic as it has been traditionally
understood as studied'' will be referred to as {\it persistent} logic
to emphasize its treatment of evidence. 

Linear logic, which was popularized by Girard \cite{girard87linear},
treats evidence as an {\it ephemeral} resource, which is generally
consumed when it is used. Linear logic, like persistent logic, comes
in classical and intuitionstic flavors. We will favor intuitionstic
linear logic in part because the propositionsions of intuitionstic
linear logic (written $A$, $B$, $C$, \ldots) have a more natural
correspondance with physical resources. Linear conjunction $A \tensor
B$ represents the resource built from the resources $A$ and $B$; if
you have both soup {\it and} a sandwitch, that resource can be
represented by the proposition ${\sf soup} \otimes {\sf
  sandwich}$. Linear implication $A \lolli B$ represents a resource
that can interact with another resource $A$ to produce a resource
$B$. A robot with batteries not included could be represented as the
resource ${\sf batteries} \lolli {\sf robot}$, and the linear resource
${\sf dollars} \lolli {\sf soup} \tensor {\sf sandwich}$ represents
the ability to use some dollars to obtain lunch but only
once.\footnote{Conjunction will always bind more tightly than
  implication, so this is equivalent to the proposition ${\sf dollars}
  \lolli ({\sf soup} \tensor {\sf sandwich})$.} Linear logic also has
a modality ${!}A$ representing a resource that can be used to generate
any number of $A$ resources, including zero. The Panera ``You Pick
Two'' menu might be represented as
\[ {!}({\sf dollars} \lolli {\sf soup} \tensor {\sf sandwich}) \otimes
{!}({\sf dollars} \lolli {\sf soup} \tensor {\sf salad}) \otimes
{!}({\sf dollars} \lolli {\sf sandwich} \tensor {\sf salad}),\] as the
menu gives you the opportunity to trade dollars for two of the three
components any number of times.

\begin{figure}[t]
\begin{tabbing}
\quad $A$ \,\, \=  
   $::= p \mid {!}A \mid \one \mid A \tensor B \mid A \lolli B$\\
\quad $\Gamma$ \> $::= \cdot \mid \Gamma, A$ \qquad \= {\it (multiset)}\\
\quad $\Delta$ \> $::= \cdot \mid \Delta, A$ \> {\it (multiset)}\\
\end{tabbing}
%
%
\quad \fbox{$\seq{\Gamma}{\Delta}{A}$}
\[
\infer[{\it id}]
{\seq{\Gamma}{p}{p}}
{}
\qquad
\infer[{\it copy}]
{\seq{\Gamma, A}{\Delta}{C}}
{\seq{\Gamma, A}{\Delta, A}{C}}
%
\]

\[
%
\infer[{!}_R]
{\seq{\Gamma}{\cdot}{{!}A}}
{\seq{\Gamma}{\cdot}{A}}
\qquad
\infer[{!}_L]
{\seq{\Gamma}{\Delta, {!}A}{C}}
{\seq{\Gamma, A}{\Delta}{C}}
\qquad
\infer[\one_R]
{\seq{\Gamma}{\cdot}{\one}}
{}
\qquad
\infer[\one_L]
{\seq{\Gamma}{\Delta, \one}{C}}
{\seq{\Gamma}{\Delta}{C}}
\]

\[
%
\infer[{\tensor}_R]
{\seq{\Gamma}{\Delta_1,\Delta_2}{A \tensor B}}
{\seq{\Gamma}{\Delta_1}{A}
 &
 \seq{\Gamma}{\Delta_2}{B}}
\qquad
\infer[{\tensor}_L]
{\seq{\Gamma}{\Delta, A \tensor B}{C}}
{\seq{\Gamma}{\Delta, A, B}{C}}
\]

\[
%
\infer[{\lolli}_R]
{\seq{\Gamma}{\Delta}{A \lolli B}}
{\seq{\Gamma}{\Delta, A}{B}}
\qquad
\infer[{\lolli}_L]
{\seq{\Gamma}{\Delta_1,\Delta_2, A \lolli B}{C}}
{\seq{\Gamma}{\Delta_1}{A}
 &
 \seq{\Gamma}{\Delta_2, B}{C}}
%
\]
\caption{Intuitionstic linear logic}
\label{fig:linear}
\end{figure}





in classical and intuitionstic varieties; 

varities) the truth of a proposition is a {\it persistent resource} --
if we have evidence 

Another name that would work
is {\it structural} logic. In formal presentations of logic, this
persistence manifests itself as the so-called {\it structural
  properties} of hypothetical reasoning -- {\it weakening} (which
allows hypotheses to go unused), {\it contraction} (which allows
hypotheses to be duplicated for reuse), and {\it exchange} (which
enforces that the ordering of hypotheses is not meaningful)
\cite{gentzen35untersuchungen}.

Linear logic, popularized by Girard \cite{girard87linear}, is a rich
logic with an inherent notion of state and stateful change. This
natural expression of stateful change makes linear logic a plausible
basis for a logical framework for the formalization of stateful
systems. 


The proof theory of linear logic -- both the classical linear logic
favored by Girard and the intuitionstic variant -- are by now
relatively standard. The presentation of the so-called {\it
  multiplicative, exponential} fragment of intuitionstic linear logic
(or {\it MELL}) in Figure~\ref{fig:linear} corresponds to Andreoli's
dyadic system \cite{andreoli92logic}, Barber's dual intuitionstic
linear logic \cite{barber96dual}, and Chang et al.'s judgmental
analysis of intuitionstic linear logic \cite{chang03judgmental}.  The
only oddity in this presentation is that there are two separate
categories of atomic proposition, named {\it positive} ($p^+$) and
{\it negative} ($p^-$); the two initial rules ${\it init}^+$ and ${\it
  init}^-$ demand that both are treated the same way.

Linear logic, in both its intuitionstic and classical variants, has
many uses throughout logic and computer science. While this thesis
will make some general observations about, and contributions to, the
proof theory of intuitionstic substructural logic,\footnote{Linear logic is
just one of the many substructural logics, a point we will return to
later.} I do so in pursuit of a very specific goal: I want to {\it encode
stateful evolving systems in substructural logic.}

\section{Focused logic and synthetic inference rules}


 This can be phrased
in one of two ways: we can say that there are now three judgment
forms:
\begin{itemize}
\item $\mildrfoc{\Gamma}{\Delta}{C}$ (the {\it right focus} judgment),
\item $\mildinv{\Gamma}{\Delta}{C}$ (the {\it inversion} judgment), and
\item $\mildlfoc{\Gamma}{\Delta}{A}{C}$ (the {\it left focus} judgment).
\end{itemize}

\[
\infer[{\it focus}R]
{\seq{\Gamma}{\Delta}{A^+}}
{\seq{\Gamma}{\Delta}{[A^+]}}
\qquad
\infer[{\it focus}L]
{\seq{\Gamma}{\Delta,A^-}{C}}
{\seq{\Gamma}{\Delta,[A^-]}{C}}
\]\[
\infer[{\it blur}R]
{\seq{\Gamma}{\Delta}{[A^-]}}
{\seq{\Gamma}{\Delta}{A^-}}
\qquad
\infer[{\it blur}L]
{\seq{\Gamma}{\Delta, [A^+]}{C}}
{\seq{\Gamma}{\Delta, A^+}{C}}
\]


The story of {\it focusing} in linear logic is not quite so 
obviously settled, however, especially when it comes to the treatment
of atoms. To try and explan why, we will first consider at 
an intuitionstic system that is, I believe, faithfully adapted from 
Andreoli's original presentation of linear logic; this system is
presented in Figure~\ref{fig:focused}.

\section{Positive atomic propositions}