\chapter{Substructural logical specifications}
\label{chapter-framework}

Logical framework time!

Example of active (or higher-order) rewriting rules
\begin{align*}
x_1{:}\susp{{\sf p2}({\sf c})}, ~~
x_2{:}\susp{{\sf p1}({\sf c})}, ~~
x_3{:}\istrue{(\forall x.\,{\sf p_1}(x) 
                \lefti \{ {\sf p_2}(x) \lefti \{ {\sf p_3}(x) \} \})}, ~~
x_4{:}\istrue{({\sf p3}({\sf c}) \lefti \{ {\sf p_4} \})} & \\
\leadsto ~~~ 
x_1{:}\susp{{\sf p2}({\sf c})}, ~~
x_5{:}\istrue{({\sf p_2}({\sf c}) \lefti \{ {\sf p_3}({\sf c}) \})}, ~~
x_4{:}\istrue{({\sf p3}({\sf c}) \lefti \{ {\sf p_4} \})} & \\
\leadsto ~~~ 
x_6{:}\susp{{\sf p3}({\sf c})}, ~~
x_4{:}\istrue{({\sf p3}({\sf c}) \lefti \{ {\sf p_4} \})} & \\
\leadsto ~~~ 
x_7{:}\susp{{\sf p_4}} & \\
\end{align*}


\section{Carving out a logical framework}

\subsection{Canonical LF as a term language}
\label{sec:why-not-fully-dependent}

Talk about Jason's thesis \cite{}. Talk about the pain of concurrent 
equality \cite{}.

\subsection{Deductive terms}

\subsection{Concurrent traces}
\label{sec:framework-substprop}


(Talk about the notation $\tackon{\Theta}{C^+}$, there's a backwards
reference to this section)

\subsection{Concurrent equality}

\subsection{Concurrent equality}
\label{sec:linconcurrenteq}

Concurrent equality is a notion of equivalence that operates on
synthetic derivations.  It represents an intermediate point between
focusing and multifocusing \cite{chaudhuri08canonical}.  Consider the
sequent in focused linear logic:
\[
\mildseq{a^+ \lolli {\uparrow}(b^+ \otimes c^+), ~
  b^+ \lolli {\uparrow}d^+, ~
  c^+ \lolli {\uparrow}e^+, ~
  d^+ \otimes e^+ \lolli {\uparrow}f^+ ~~}
  {~~
  \langle a^+ \rangle
  ~~}
  {~~f^+}
\]
Let $\Gamma = \left(a^+ \lolli {\uparrow}(b^+ \otimes c^+), ~
  b^+ \lolli {\uparrow}d^+, ~
  c^+ \lolli {\uparrow}e^+, ~
  d^+ \otimes e^+ \lolli {\uparrow}f^+ \right)$.
There are two different focused derivations of this
sequent: the one that transitions $\langle b^+ \rangle$ to $\langle
d^+ \rangle$ first, and the one that transitions 
$\langle c^+ \rangle$ to $\langle e^+ \rangle$ first:
\[
\infer
{\mildseq{\Gamma}{\langle a^+ \rangle}{f^+}}
{\infer
{\mildseq{\Gamma}{\langle b^+ \rangle, \langle c^+ \rangle}{f^+}}
{\infer
{\mildseq{\Gamma}{\langle d^+ \rangle, \langle c^+ \rangle}{f^+}}
{\infer
{\mildseq{\Gamma}{\langle d^+ \rangle, \langle e^+ \rangle}{f^+}}
{\infer
{\mildseq{\Gamma}{\langle f^+ \rangle}{f^+}}
{}}}}}
\qquad
\deduce
{\mathstrut}
{\deduce
{\mathstrut}
{\deduce
{\mathstrut}
{\mbox{\it vs.}}}}
\qquad
\infer
{\mildseq{\Gamma}{\langle a^+ \rangle}{f^+}}
{\infer
{\mildseq{\Gamma}{\langle b^+ \rangle, \langle c^+ \rangle}{f^+}}
{\infer
{\mildseq{\Gamma}{\langle b^+ \rangle, \langle e^+ \rangle}{f^+}}
{\infer
{\mildseq{\Gamma}{\langle d^+ \rangle, \langle e^+ \rangle}{f^+}}
{\infer
{\mildseq{\Gamma}{\langle f^+ \rangle}{f^+}}
{}}}}}
\]
If we think about these two proofs in terms of the series of
transitions they embody, it's not so clear we want to think of them as
different. In both cases, there is an $a^+$ resource that transitions
to a $b^+$ resource and a $c^+$ resource, and then $b^+$ transitions
to $d^+$ while, independently, the $c^+$ transitions to $e^+$. Then,
finally, the $d^+$ and $e^+$ combine to transition to $f^+$, which
completes the trace. The independence here is key: if two focusing
phases consume different resources and both end focus with
${\uparrow}_L$ (as opposed to ${\it id}^-$), then we can treat them as
independent and concurrent steps in the process of proving the same
right-hand side. {\it Concurrent equality} is the equivalence relation
on focused proofs that treats all proofs that differ only in the
interleaving of independent and concurrent steps as equal.  This
equivalence relation was used in the definition of CLF
\cite{watkins02concurrent}, but in a greatly restricted way that will
be reflected in Chapter 4.

Concurrent equality gives rise to an equivalence relation on focused
derivations. This equivalence relation is related to the equivalence
relation induced by {\it multifocusing}
\cite{chaudhuri08canonical}. Multifocusing is a concept that has only
been carefully explored in classical linear logic; the central change
is that the rules which begins a focusing phase (in our presentation
of MELL there were three: ${\it focus_L}$, ${\it focus_R}$, and ${\it
  copy}$) are allowed to simultaneously pull other propositions into
focus.  As an illustration, if we reuse our notation from
Section~\ref{sec:linnote} we can present the following plausible
candidates for the multifocus rules in an intuitionistic system:
\[
\infer[{\it focus}_L]
{\mildseq{\Gamma}{\Delta / A_1^-, \ldots, A_n^- }{U}}
{n > 1
 &
 \mildseq{\Gamma}{\Delta, [A_1^-], \ldots, [A_n^-]}{U}}
\quad
\infer[{\it focus}_R]
{\mildseq{\Gamma}{\Delta / A_1^-, \ldots, A_n^-}{C^+}}
{n \geq 1
 &
 \mildseq{\Gamma}{\Delta, [A_1^-], \ldots, [A_n^-]}{[C^+]}}
\]
Multifocusing, however,
appears to provide an even coarser notion of equivalence on focused
proofs than concurrent equality does. In particular, the two
distinct focusing proofs below are not concurrently equal: the proof
on the right succeeds at proving $\langle c^- \rangle$ in one step,
but leaves a subgoal in which $b^+$ is proved indirectly, whereas the
proof at the right first transitions from having $\langle a^+ \rangle$
and $a^+ \lolli {\uparrow} b^+$ resources to having a $\langle b^+
\rangle$ resource, and only then proves $\langle c^- \rangle$, leaving
a subgoal in which $b^+$ is proved directly.
\[
\infer
{\mildseq{\cdot}
  {~~
   \langle a^+ \rangle, ~
   a^+ \lolli {\uparrow}b^+, ~
   {\downarrow}{\uparrow}b^+ \lolli c^-
   ~~}
  {~~\langle c^- \rangle}}
{\infer
{\mildseq{\cdot}
  {~~
   \langle a^+ \rangle, ~
   a^+ \lolli {\uparrow}b^+
   ~~}
  {b^+}}
{\infer
{\mildseq{\cdot}
  {~~
   \langle b^+ \rangle
   ~~}
  {b^+}}
{}}}
\deduce{\mathstrut}
{\deduce{\mathstrut}
{\mbox{\it vs.}\mathstrut}}
\infer
{\mildseq{\cdot}
  {~~
   \langle a^+ \rangle, ~
   a^+ \lolli {\uparrow}b^+, ~
   {\downarrow}{\uparrow}b^+ \lolli c^-
   ~~}
  {~~\langle c^- \rangle}}
{\infer
{\mildseq{\cdot}
  {~~
   \langle b^+ \rangle, ~
   {\downarrow}{\uparrow}b^+ \lolli c^-
   ~~}
  {~~\langle c^- \rangle}}
{\infer
{\mildseq{\cdot}
  {~~
   \langle b^+ \rangle
   ~~}
  {~~b^+}}
{}}}
\]
Despite the lack of a full account of intuitionistic multifocusing, we
can observe that the analogue of this sequent in classical linear
logic has only one multifocused proof, and it is reasonable to
conjecture that an account of multifocusing for intuitionistic logic
would also relate these proofs. In classical linear logic,
multifocusing offers a very fundamental normal form: any two proofs
that can be made equal by locally permuting inference rules have the
same multifocused proof.

CLF's restricted form of concurrent equality will be sufficient for
the logical framework in Chapter 4. In fact, for the fragment of the
the logic in Chapter 3 that comprises our logical framework in Chapter
4, I conjecture that concurrent equality and the equality given by
multifocusing coincide.\footnote{This obviously means that the example
  above will be outside the logical fragment that comprises the logical
  framework.}  This conjecture is obviously difficult to make precise,
much less prove, without a general theory of multifocusing in
intuitionistic logic.


\subsection{A warning about normalization}
\label{sec:warning}

In our earlier discussion of hereditary substitution and canonical
forms in Section~\ref{sec:linlogicalframeworks}, we mentioned that the
normalization theorem provided by hereditary substitution was weaker
than the so-called weak normalization theorem for LF. That is because
the weak normalization theorem says that any well-typed term can be
converted into a canonical ($\beta$-normal and $\eta$-long) term by a
particular series of $\beta$ and $\eta$ conversions. It is
self-evident, by this statement of the theorem, that the resulting
canonical term is equivalent to the original term.

On the other hand, when we use hereditary substitution in the obvious
way to obtain a Canonical LF term from an arbitrary non-canonical LF
term, we gain {\it no guarantees} about the relationship between the
non-canonical LF term and the Canonical LF term. The statement of the
theorem does not preclude taking a $\beta$-normal, $\eta$-long LF term
(like $\lambda x. \lambda y. x$ of type $p \rightarrow p \rightarrow
p$ for some atomic type $p$) into a structurally different Canonical
LF term (like $\lambda x. \lambda y. y$, which also has type $p
\rightarrow p \rightarrow p$). It is possible to gain such a guarantee
for LF, as Martens and Crary have shown in unpublished work
\cite{martens11mechanizing}, but this result is a non-trivial statement
about the constructive content of the normalization theorem. 

In our setting, we should be concerned that we might take a focused
proof, turn it into an unfocused proof by the obvious de-focalization
procedure (the constructive content of
Theorem~\ref{thm:linfocsound}), and then turn it back into a focused
proof by focalization (the constructive content of
Theorem~\ref{thm:linfoccomplete}) only to obtain a proof that was not
identical or even related. This is not at all a merely hypothetical
concern. We can run the mechanized structural focalization result from
\cite{simmons11structural} on a persistent proposition,
%
   $a^+ \supset 
   {\downarrow}(a^+ \supset {\uparrow}b^+) \supset
   {\downarrow}({\downarrow}{\uparrow}b^+ \supset c^-) \supset
   c^-$, 
%
which is similar to the example from
Section~\ref{sec:linconcurrenteq}.  In persistent logic (as in
linear logic) that proposition has two focused propositions that
are probably multifocusing equivalent (given a reasonable intuitionistic
notion of multifocusing) but that are not concurrently equivalent
under the proposed definition of concurrent equality. 
However, if we take the focused proof that focuses 
first on $a^+ \supset {\uparrow}b^+$, transform it into an unfocused 
proof, and then re-focus it, we will get the proof that focuses 
first on ${\downarrow}{\uparrow}b^+ \supset c^-$. Focalization,
in other words, is not a partial inverse of de-focalization in the structural
focalization development, except maybe modulo the (as yet undefined)
equivalence relation established by multifocusing. 

This example illustrates why we must be careful, but it is not a fatal
flaw for two reasons. The first reason is the aforementioned
conjecture that, for the restricted logical fragment defined in
Chapter 4 as the basis of our logical framework, the focalizations of
two proofs are concurrently equal if and only if the original proofs
are convertible by local permutations of rules, the same condition
that multifocusing satisfies. If this conjecture holds, it ought to be
the case that, modulo this coarser equivalence, focalization {\it is}
a partial inverse of de-focalization. Second, what is really at stake
here is our ability to write down non-normal proofs in a logical
framework that then normalizes them -- which is what the Twelf
implementation of LF and the Celf implementation of CLF do -- with the
confidence that we can look at a non-normal proof and know its
corresponding canonical form. In this thesis, we will be content to
work throughout with focused proofs and their analogues, so we can
afford to leave questions about convertability and weak normalization
to future work.


\subsection{Pseudo-positive atoms}
\label{sec:pseudopositive}

\section{Logic programming interpretation}
\label{sec:framework-logicprog}

In Chapter 3 I call this distinction one between ``concurrent and deductive''
proofs, operationally it's the difference between forward chaining 
and backward chaining. Maintaining the distinction between these is why
we don't want the class $p^-_\mlax$ in the logic.

We have talked about {\it concurrent} and {\it deductive} proof
objects, and also about the intuitive notion of {\it concurrent
  computation} as the non-backtracking, forward-chaining 

(definitely discuss forward-chaining and backward-chaining as concepts)

Expand on literature review of quiescence from HOSC Section 4

Leave the question of quescense versus eagerly-trying-to-right-focus
versus saturation ambiguous. If you talk about pure saturation the
forward-reference Chapter 8. 

\subsection{Modes and well-moded specifications}
\label{sec:framework-modes}

\section{Adequate encoding}

The lambda
cacluslus usually has application $e_1\,e_2$ (encoded in LF as ${\sf
  app}\,\interp{e_1}\,\interp{e_2}$) and abstraction $\lambda x.e$
(encoded in LF as ${\sf lam}\,\lambda x. \interp{e}$). 

\subsection{Adequacy for LF and deductive terms}

\subsection{Adequacy for concurrent traces}

