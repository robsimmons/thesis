\documentclass[12pt]{article}
\usepackage{latexsym}
\usepackage{amssymb}            % for \multimap (-o)
\usepackage{stmaryrd}           % for \binampersand (&), \bindnasrepma (\paar)

% symbols of linear logic
\newcommand{\lolli}{\multimap}
\newcommand{\tensor}{\otimes}
\newcommand{\with}{\mathbin{\binampersand}}
\newcommand{\paar}{\mathbin{\bindnasrepma}}
\newcommand{\one}{\mathbf{1}}
\newcommand{\zero}{\mathbf{0}}
\newcommand{\bang}{{!}}
\newcommand{\whynot}{{?}}
\newcommand{\bilolli}{\mathrel{\raisebox{1pt}{\ensuremath{\scriptstyle\circ}}{\lolli}}}
% \oplus, \top, \bot

% judgments of linear logic
\newcommand{\seq}[3]{{#1};{#2} \longrightarrow {#3} \mathstrut}

\newcommand{\mildrfoc}[3]{{#1};{#2} \longrightarrow [{#3}] \mathstrut}
\newcommand{\mildinv}[3]{{#1};{#2} \longrightarrow {#3} \mathstrut}
\newcommand{\mildlfoc}[4]{{#1};{#2}, [{#3}] \longrightarrow {#4} \mathstrut}


\newcommand{\urfoc}[3]{{#1};{#2} \longrightarrow [{#3}] \mathstrut}
\newcommand{\ulfoc}[4]{{#1};{#2} \,[#3] \longrightarrow {#4} \mathstrut}
\newcommand{\uinv}[4]{{#1};{#2};{#3} \longrightarrow {#4} \mathstrut}

\newcommand{\stableR}[1]{{#1}\,\mathit{stable_R} \mathstrut}
\newcommand{\stableL}[1]{{#1}\,\mathit{stable_L} \mathstrut}


\begin{document}
I think I've sorted out the problems that I was having. However, in doing
so it's been very important to make some stability assumptions in the 
cut admissibility proof. In particular, in these two cases:
\[
\infer-
{\mildseq{\Gamma}{\Delta', \delta}{U}}
{\mildseq{\Gamma}{\delta}{A^+}
 &
 \mildseq{\Gamma}{\Delta', A^+}{U}}
\quad
\infer-
{\mildseq{\Gamma}{\delta', \Delta}{\gamma}}
{\mildseq{\Gamma}{\Delta}{A^-}
 &
 \mildseq{\Gamma}{\delta', A^-}{\gamma}}
\]
I believe that it is rather important, if a structural proof is our aim, to
force $\Gamma'$, $\Gamma$, and $U$ to be stable (contexts contain no positive
propositions and succeedents are not negative propositions).

To see why, try running through what a structural focalization proof
{\it ought} to do on these cases:
\[
\infer-
{\mildseq{\Gamma}{A \otimes B, \langle p^+ \rangle, p^+ \lolli q^+}
  {C \multimap D}}
{\infer
 {\mildseq{\Gamma}{\langle p^+ \rangle, p^+ \lolli q^+}{q^+}}
 {\infer
  {\mildseq{\Gamma}{\langle p^+ \rangle, [p^+ \lolli q^+]}{q^+}}
  {\infer
   {\mildseq{\Gamma}{\langle p^+ \rangle}{[p^+]}}
   {}
   &
   \infer
   {\mildseq{\Gamma}{[q^+]}{q^+}}
   {\infer
    {\mildseq{\Gamma}{q^+}{q^+}}
    {\infer
     {\mildseq{\Gamma}{\langle q^+ \rangle}{q^+}}
     {\infer
      {\mildseq{\Gamma}{\langle q^+ \rangle}{[q^+]}}
      {}}}}}}
 &
 \infer
 {\mildseq{\Gamma}{A \otimes B, q^+}{C \multimap D}}
 {\infer
  {\mildseq{\Gamma}{A \otimes B, \langle q^+ \rangle}{C \lolli D}}
  {\infer
   {\mildseq{\Gamma}{A, B, \langle q^+ \rangle}{C \lolli D}}
   {\deduce
    {\mildseq{\Gamma}{A, B, \langle q^+ \rangle, C}{D}}
    {\vdots}}}}}
\]
\[
\infer-
{\mildseq{\Gamma}{A \otimes B, a^- \lolli b^-}{\langle b^- \rangle}}
{\infer
 {\mildseq{\Gamma}{A \otimes B}{a^-}}
 {\infer
  {\mildseq{\Gamma}{A \otimes B}{\langle a^- \rangle}}
  {\deduce
   {\mildseq{\Gamma}{A, B}{\langle a^- \rangle}}
   {\vdots}}}
 &
 \infer
 {\mildseq{\Gamma}{a^-, a^- \lolli b^-}{\langle b^- \rangle}}
 {\infer
  {\mildseq{\Gamma}{a^-, [a^- \lolli b^-]}{\langle b^- \rangle}}
  {\infer
   {\mildseq{\Gamma}{a^-}{[a^-]}}
   {\infer
    {\mildseq{\Gamma}{a^-}{a^-}}
    {\infer
     {\mildseq{\Gamma}{a^-}{\langle a^- \rangle}}
     {\infer
      {\mildseq{\Gamma}{[a^-]}{\langle a^- \rangle}}
      {}}}}
   &
   \infer
   {\mildseq{\Gamma}{[b^-]}{\langle b^- \rangle}}
   {}}}}
\]
I claim that the answer is
``permute the leftmost proof {\it around} the blocking invertible step
until the principle formula is the only invertible formula, then
begin a principal cut.'' I also claim that formalizing this has quadratic
proof complexity. 

\paragraph{Solution 1} Restrict cut as described above. This is my
desired solution, assuming the rest of the proof goes through. However,
your shiftless unfocused admissibility lemmas seem to
depend on the non-stable cut principles, so we're back to a situation
where you need to use shifts (or Miller-style delay operators)
to prove unfocused admissibility with linear proof complexity.

\paragraph{Solution 2} You could factor out all that complexity
if you factored all your left rules into one forward-chaining monotonicity 
rule
\[
\infer
{\mildseq{\Gamma}{\Delta, A^+}{U}}
{A^+ \leadsto \Theta
 &
 \forall (\Delta' \in \Theta)
 &
 \longrightarrow 
 &
 \mildseq{\Gamma}{\Delta, \Delta'}{U}}
\]
but I don't plan to take that route right now.

\end{document}


There is a still-deeper problem with the structural proof of cut admissibility
in the confluent system. In this presentation, I use ``unstable'' to refer
to an inversion sequent with multiple 

\paragraph{Claim 1}
In the positive focused cut, we {\it must} allow $\Delta'$ to be unstable.
\[
\infer-[]
{\Gamma; \Delta', \Delta \vdash U}
{\Gamma; \Delta \vdash [A^+]
 &
 \Gamma; \Delta', A^+ \vdash U}
\]

\paragraph{Justification 1}
If we do not allow $\Delta'$ to be unstable, we can't invoke the
induction hypothesis in a way that allows us to prove the principal
cut for tensor.
\[
\infer-[]
{\Gamma; \Delta', \Delta_1, \Delta_2 \vdash U}
{\infer
 {\Gamma; \Delta_1, \Delta_2 \vdash [A^+ \otimes B^+]}
 {\Gamma; \Delta_1 \vdash [A^+] 
  &
  \Gamma; \Delta_2 \vdash [B^+]}
 &
 \infer
 {\Gamma; \Delta', A^+ \otimes B^+ \vdash U}
 {\Gamma; \Delta', A^+, B^+ \vdash U}}
\]
Reduces either as
\[
\infer-[]
{\Gamma; \Delta', \Delta_1, \Delta_2 \vdash U}
{\Gamma; \Delta_1 \vdash [A^+] 
 &
 \infer-[]
 {\Gamma; \Delta', A^+, \Delta_2 \vdash U}
 {\Gamma; \Delta_2 \vdash [B^+]
  &
  \Gamma; \Delta', A^+, B^+ \vdash U}}
\]
or as
\[
\infer-[]
{\Gamma; \Delta', \Delta_1, \Delta_2 \vdash U}
{\Gamma; \Delta_2 \vdash [B^+] 
 &
 \infer-[]
 {\Gamma; \Delta', \Delta_1, B^+ \vdash U}
 {\Gamma; \Delta_1 \vdash [A^+]
  &
  \Gamma; \Delta', A^+, B^+ \vdash U}}
\]
and the ``more upper'' cut has an independent positive in the context
in both cases ($A^+$ in the first case, and $B^-$ in the second).

\paragraph{Claim 2}
In the negative unfocused cut, we {\it must} allow $\Delta'$ to be unstable.
\[
\infer-[]
{\Gamma; \Delta', \Delta \vdash U}
{\Gamma; \Delta \vdash A^-
 &
 \Gamma; \Delta', A^- \vdash U}
\]

\paragraph{Justification 2}
This cut arises inevitably from the principal cut for ${\downarrow}A^-$
(or blur focus, if you don't have shifts), which is an instance of positive
focused cut, where we have already established that we must have
an unstable $\Delta'$ (Claim 1).
\[
\infer-[]
{\Gamma; \Delta', \Delta \vdash U}
{\infer[]
 {\Gamma; \Delta \vdash [{\downarrow}A^-]}
 {\Gamma; \Delta \vdash A^-}
 &
 \infer[]
 {\Gamma; \Delta', {\downarrow}A^- \vdash U}
 {\Gamma; \Delta', A^- \vdash U}}
\]

\paragraph{Claim 1}
In the positive un-focused cut, we
{\it must} allow $\Delta'$ to be unstable. Otherwise, we are 



